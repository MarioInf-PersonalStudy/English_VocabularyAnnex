% tables/table_irregulars.tex

% Aquí definimos cómo debe imprimirse cada fila
\newcommand{\newIrregularVerb}[8]{%
    \eng{#1}
    \ifthenelse{\equal{#2}{}}
        {}
        {(\emph{US:\usa{#2}})}
    &
    \eng{#3}
    \ifthenelse{\equal{#4}{}}
        {}
        {(\emph{US:\usa{#4}})}
    &
    \eng{#5}
    \ifthenelse{\equal{#6}{}}
        {}
        {(\emph{US:\usa{#6}})}
    &
    \esp{#7}
    \ifthenelse{\equal{#8}{}}
        {}
        {(\emph{Ctx:\esp{#8}})}
    \\
    \hline
}


% Macro para imprimir la tabla completa
\newcommand{\PrintIrregularVerbs}[2]{%
  \begingroup
  \renewcommand{\arraystretch}{1}
  \setlength{\tabcolsep}{3pt}
  \centering

  \begin{longtable}{|p{0.22\textwidth}|p{0.22\textwidth}|p{0.22\textwidth}|p{0.25\textwidth}|}
    \hline
    \rowcolor{Color_TitleTableBackground}\multicolumn{4}{|c|}{\LARGE\texttt{\textbf{#2}}} \\
    \hline
    \rowcolor{Color_TitleTableBackground}\textbf{Present} & \textbf{Past Simple} & \textbf{Past Participle} & \textbf{Spanish} \\
    \hline
    \endfirsthead

    \hline
    \endhead

    \hline
    \endfoot

    \hline
    \endlastfoot

    \input{#1}
  \end{longtable}

  \endgroup
}
