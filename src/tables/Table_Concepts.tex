% tables/table_concepts.tex

% Aquí definimos cómo debe imprimirse cada fila
\newcommand{\newVocabularyConcept}[5]{%
    \eng{#1}
    \ifthenelse{\equal{#2}{}}
        {}
        {(\emph{US:\usa{#2}})}
    &
    \textit{#3} & 
    \esp{#4}
    \ifthenelse{\equal{#5}{}}
        {}
        {(\emph{Ctx:\esp{#5}})}
    \\ 
    \hline
}

% Esta macro imprime la tabla y carga los datos definidos
\newcommand{\PrintTableConcepts}[2]{%
    \begingroup
    \renewcommand{\arraystretch}{1}
    \setlength{\tabcolsep}{3pt}
    \centering
    
    \begin{longtable}{|p{0.45\textwidth}|c|p{0.45\textwidth}|}
        \hline
        \rowcolor{Color_TitleTableBackground}\multicolumn{3}{|c|}{\LARGE\texttt{\textbf{#2}}} \\
        \hline
        \rowcolor{Color_TitleTableBackground}\textbf{English} & \textbf{Type} & \textbf{Spanish} \\
        \hline
        \endfirsthead

        \hline
        \endhead
        
        \hline
        \endfoot

        \hline
        \endlastfoot

        \input{#1}
    \end{longtable}

  \endgroup
}
