% tables/table_concepts.tex

% Redefine la macro que usaremos para introducir vocabulario.
% Esta definición NO imprime nada directamente cuando se llama desde data_food
\newcommand{\newVocabularyConcept}[3]{%
  \VocabularyEntry{#1}{#2}{#3}%
}

% Aquí definimos cómo debe imprimirse cada fila
\newcommand{\VocabularyEntry}[3]{%
  \eng{#1} & \textit{#2} & \esp{#3} \\
  \hline
}

% Esta macro imprime la tabla y carga los datos definidos
\newcommand{\PrintTableConcepts}[1]{%
  \begingroup
  \renewcommand{\arraystretch}{1.3}
  \begin{center}
    \begin{tabular}{|p{0.3\textwidth}|p{0.2\textwidth}|p{0.4\textwidth}|}
      \hline
      \textbf{English} & \textbf{Type} & \textbf{Spanish} \\
      \hline
      \input{#1}
    \end{tabular}
  \end{center}
  \endgroup
}
