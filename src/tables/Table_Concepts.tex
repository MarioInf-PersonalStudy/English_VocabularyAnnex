% tables/table_concepts.tex

% Redefine la macro que usaremos para introducir vocabulario.
% Esta definición NO imprime nada directamente cuando se llama desde data_food
\newcommand{\newVocabularyConcept}[3]{%
  \VocabularyEntry{#1}{#2}{#3}
}

% Aquí definimos cómo debe imprimirse cada fila
\newcommand{\VocabularyEntry}[3]{%
  \eng{#1} & \textit{#2} & \esp{#3} \\
  \hline
}

% Esta macro imprime la tabla y carga los datos definidos
\newcommand{\PrintTableConcepts}[2]{%
  \begingroup
  \renewcommand{\arraystretch}{1.2}
  \begin{center}
    
    \begin{longtable}{|p{0.42\textwidth}|c|p{0.42\textwidth}|}
        \hline
        \rowcolor{Color_TitleTableBackground}\multicolumn{3}{|c|}{\LARGE\texttt{\textbf{#2}}} \\
        \hline
        \rowcolor{Color_TitleTableBackground}\textbf{English} & \textbf{Type} & \textbf{Spanish} \\
        \hline
        \endfirsthead
        \input{#1}
    \end{longtable}

  \end{center}
  \endgroup
}
